\documentclass{beamer}
\usetheme{Boadilla}
\usepackage[utf8x]{inputenc}
\usepackage{listings}
\title{LS IoT Platform}
\subtitle{Piattaforma per il monitoraggio di macchine utensili con integrazione a software ERP Microsoft Dynamics NAV}
\author{Vincenzo Nucci e Matteo Tiberi}
%\author{Matteo Tiberi}
\institute{Università di Camerino}

\begin{document}
	\begin{frame}
	\titlepage
\end{frame}

\begin{frame}
\frametitle{LS IoT Platform}
\begin{itemize}
	\item espone servizi REST per letture di sensori
	\begin{itemize}
		\item ultima lettura di un sensore
		\item letture di una certa settimana
		\item letture di un certo mese
		\item letture di determinati campi dei sensori
	\end{itemize}
	\item servizio di sottoscrizione con notifiche PUSH
	\begin{itemize}
		\item debolmente accoppiato grazie ad Apache ActiveMQ
	\end{itemize}
	\item integrazione della sottoscrizione con Microsoft Dynamics NAV
	\begin{itemize}
		\item utilizzo dei web services SOAP offerti da NAV
	\end{itemize}
	\item indipendente da sorgenti dati e formato dei dati
	\begin{itemize}
		\item grazie alle interfacce e Apache Avro
	\end{itemize}
	\item interfaccia web
	\begin{itemize}
		\item pagina per la registrazione delle applicaizoni
		\item pagina per la gestione delle richieste
		\item pagina per la gestione dei servizi attivi per gli utenti
	\end{itemize}
\end{itemize}
\end{frame}

\begin{frame}
\frametitle{Pagina web di richiesta token}
\includegraphics[width=1\textwidth]{images/RequestPagePlatform.png}
\end{frame}

\begin{frame}
\frametitle{Richieste database}
\includegraphics[width=1\textwidth]{images/DBPlatform2.png}
\end{frame}

\begin{frame}
\frametitle{Pagina web gestione delle richieste}
\includegraphics[width=1\textwidth]{images/managePagePlatform.png}
\end{frame}

\begin{frame}
\frametitle{Sottoscrizioni database}
\includegraphics[width=1\textwidth]{images/DBPlatform3.png}
\end{frame}

\begin{frame}
\frametitle{Pagina web gestione dei servizi utenti}
\includegraphics[width=1\textwidth]{images/UserSubscriptionsPlatform.png}
\end{frame}

\begin{frame}
\frametitle{Esempio di un servizio}
\includegraphics[width=1\textwidth]{images/getlastmeasure.png}
\end{frame}

\begin{frame}
\frametitle{Risposta chiamata getlastmeasure}
\includegraphics[width=1\textwidth]{images/Postman1.png}
\end{frame}

\begin{frame}
\frametitle{Smart Object Page}
\includegraphics[width=1\textwidth]{images/SmartObjectsPlatform.png}
\end{frame}

\begin{frame}
\frametitle{Subscribe Rule}
\begin{itemize}
\item messaggio json da inviare per usare il servizio push
\item l'utente specifica la condizione
\begin{itemize}
\item nome e IP applicazione
\item coda o topic activemq
\item subscribe rule
\begin{itemize}
\item nome tabella da interrogare
\item lista dei campi da monitorarne i cambiamenti
\item espressione CRON
\item clausola where

\end{itemize}
\end{itemize}
\end{itemize}
\end{frame}

\begin{frame}
\frametitle{Clausola where}
\begin{itemize}
\item condition
\begin{itemize}
\item insieme di oggetti avro annidati
\end{itemize}
\item formula
\begin{itemize}
\item semplice stringa (per NAV)
\end{itemize}
\end{itemize}
\end{frame}

\begin{frame}
\frametitle{Albero della condition}
\includegraphics[width=1\textwidth]{images/struttura-query-tree.png}
\end{frame}

\begin{frame}
\frametitle{Subscriberule di una applicazione}
\includegraphics[width=0.6\textwidth]{images/subscribe-json-1.png}
\end{frame}

\begin{frame}
\frametitle{Regole nel database}
\includegraphics[width=1\textwidth]{images/DBPlatform1.png}
\end{frame}

\begin{frame}
\frametitle{Schema JSON Subscribe di NAV}
\includegraphics[width=0.6\textwidth]{images/subscribe-json-2.png}
\end{frame}

\begin{frame}
\frametitle{Class Diagram Subscribe Rule Interface}
\includegraphics[width=1\textwidth]{images/ClassDiagram7.png}
\end{frame}

\begin{frame}
\frametitle{Class Diagram RetrievedDataInterface 1}
\includegraphics[width=1\textwidth]{images/ClassDiagram2.png}
\end{frame}

\begin{frame}
\frametitle{Class Diagram RetrievedDataInterface 2}
\includegraphics[width=1\textwidth]{images/ClassDiagram1.png}
\end{frame}

\begin{frame}
\frametitle{metodo di popolamento dati json}
\includegraphics[width=1\textwidth]{images/popolamento-json.png}
\end{frame}

\begin{frame}
\frametitle{Class Diagram AbstractConnection}
\includegraphics[width=1\textwidth]{images/main.png}
\end{frame}

%-------------------------inizio client-----------------------------
\begin{frame}
\frametitle{Client per la piattaforma}
\begin{itemize}
	\item Sviluppato su Microsoft Dynamics NAV nonostante diverse lacune dell'ambiente
	\begin{itemize}
		\item Mancata possibilità di consumo diretto di servizi REST
		\item Mancata possibilità di gestione del formato JSON
		\item Difficoltà nell'interazione con software esterni non Microsoft
	\end{itemize}
	\item Risoluzione tramite sviluppo di un client C\#
	\begin{itemize}
		\item con chiamata dei servizi REST, serializzazione e deserializzazione del JSON
		\item in conformità con le classi della piattaforma tramite Apache Avro
		\item integrato poi in NAV tramite dll 
	\end{itemize}
	\item Sviluppo di un "setup" per impostare le chiamate ai servizi su NAV
	\begin{itemize}
		\item svolto mediante 2 approcci (PLC e Machine Center)
		\item con trattamento dei dati per l'ambiente Navision
		\item evitando di prendere valori già inseriti o errati
	\end{itemize}

	\item Interazione con il servizio di sottoscrizione nell'ambiente NAV
	\begin{itemize}
		\item tramite esternazione di una codeunit come webservice SOAP
	\end{itemize}
\end{itemize}
\end{frame}
\begin{frame}
\frametitle{Client NAV}
\includegraphics[width=1\textwidth]{images/NAVClient.png}
\end{frame}

\begin{frame}
\frametitle{Class Diagram Client 1}
\includegraphics[width=0.6\textwidth]{images/ClassDiagramParte1.png}
\end{frame}

\begin{frame}
\frametitle{Class Diagram Client 2}
\includegraphics[width=0.6\textwidth]{images/ClassDiagramParte2.png}
\end{frame}

\begin{frame}
\frametitle{Ambiente di sviluppo NAV}
\includegraphics[width=1\textwidth]{images/NAVDevelopmentEnvironment.png}
\end{frame}


\begin{frame}
\frametitle{Lista delle funzioni della codeunit}
\includegraphics[width=1\textwidth]{images/NAVFunctionList.png}
\end{frame}


\begin{frame}
\frametitle{Lista PLC}
\includegraphics[width=1\textwidth]{images/PLCList.png}
\end{frame}

\begin{frame}
\frametitle{PLC Assignment List}
\includegraphics[width=1\textwidth]{images/PLCAssignmentList.png}
\end{frame}


\begin{frame}
\frametitle{Lista con i parametri}
\includegraphics[width=1\textwidth]{images/MachineParameter.png}
\end{frame}


\begin{frame}
\frametitle{PLC Reading List}
\includegraphics[width=1\textwidth]{images/PLCReadingList.png}
\end{frame}


\begin{frame}
\frametitle{Metodo SetMeasurementPLC}
\includegraphics[width=1\textwidth]{images/NAVSetMesurament.png}
\end{frame}


\begin{frame}
\frametitle{NAV servizi web}
\includegraphics[width=1\textwidth]{images/NAVServiziWeb.png}
\end{frame}


\begin{frame}
\frametitle{NAV SubscriptionPage}
\includegraphics[width=1\textwidth]{images/NAVSubscriptionPage.png}
\end{frame}

\begin{frame}
\frametitle{Metodo PushMeasurament 1}
\includegraphics[width=1\textwidth]{images/NAVPushMeasuraments1.png}
\end{frame}

\begin{frame}
\frametitle{Metodo PushMeasurament 2}
\includegraphics[width=1\textwidth]{images/NAVPushMeasuraments2.png}
\end{frame}

\begin{frame}
\frametitle{Metodo PushMeasurament 3}
\includegraphics[width=1\textwidth]{images/NAVPushMeasuraments3.png}
\end{frame}






\subsection{sub b}







\end{document}